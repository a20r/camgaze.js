
\documentclass{beamer}

\usepackage{beamerthemeshadow}

\begin{document}

\title{Camgaze.js: A JavaScript Library for Eye Tracking and Gaze Prediction}

\author[shortname]{Alex Wallar \inst{1} \and Christian Poellabauer \inst{2}
\and Aleksejs Sazonovs \inst{1} \and Patrick Flynn \inst{2}}
\institute[shortinst]{\inst{1} University of St Andrews \and \inst{2}
University of Notre Dame}

\date{\today}

\frame{\titlepage}

\frame{\frametitle{Table of contents}\tableofcontents}

\section{Introduction}

\subsection{Review}

\frame{\frametitle{What is it?}

\begin{itemize}

\item Eye tracking is a problem which tries to determine where a user is
looking on the screen \pause

\item Usually done using IR or 3D cameras \pause

\item Some webcam technologies have emerged \pause

\item However, no in-browser solutions have been presented soley using HTML5
\pause

\item Until now :)

\end{itemize}

}

\subsection{Motivation}

\frame{\frametitle{Motivation}

\begin{itemize}

\item Eye tracking can provide vital data about what is important on the screen
\pause

\item We can create more intuitive user interfaces \pause

\item Using the web, we can crowd source where people are looking at on the
website \pause

\item Also, since all of the eye tracking is done on the client side, we can
preserve user privacy

\end{itemize}

}

\subsection{Camgaze.js}

\frame{\frametitle{Camgaze.js}

\begin{itemize}

\item A library for eye tracking that is done inside a web browser using
JavaScript \pause

\item Uses only commodity camera (i.e.\ a webcam) \pause

\item Anybody can use the library without downlading any external program
besides a web browser \pause

\item It is possible to determine where the user is looking on the screen
whilst preserving user privacy and limiting server load

\end{itemize}

}

\section{Implementation}

\subsection{Overview}

\frame{\frametitle{Overview}

\begin{enumerate}

\item Obtain video from using Web RTC (Real Time Communication) library

\item Determine the positions of the pupil centroids

\item 

\end{enumerate}

}

\end{document}

